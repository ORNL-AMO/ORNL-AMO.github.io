\hypertarget{dir_fb6b2a7dc984b2196ed8d1f594cc04ee}{}\section{src/phast Directory Reference}
\label{dir_fb6b2a7dc984b2196ed8d1f594cc04ee}\index{src/phast Directory Reference@{src/phast Directory Reference}}
